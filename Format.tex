% Offset zur einfacheren Abstandsangabe (Papier links oben ist 'Null')
\setlength{\hoffset}{0cm}
\setlength{\voffset}{0cm}

% Kopfzeile
\headheight.7cm		% Hoehe vom Header

% Textfeld
\textheight23.8cm
\textwidth16cm

% Layout vertikale Abstaende
\headsep1.1cm															% Abstand Header - Text
\setlength{\topmargin}{-1in + 1.2cm}			% Abstand oberer Seitenrand - Header
\setlength{\footskip}{1.5cm}							% Abstand Text - Fusszeile

% Layout horizontale Abstaende gerade Seiten
\setlength{\evensidemargin}{-0.5cm}											% Abstand linker Seitenrand zum Textrand
% Layout horizontale Abstaende ungerade Seiten
\ifthenelse{\boolean{@twoside}}{
	\setlength{\oddsidemargin}{-\evensidemargin}						% Abstand rechter Seitenrand zum Textrand
}{}

% Notizbereiche
\setlength{\marginparwidth}{1.5cm}
\marginparsep8pt


%% Fuss- und Kopfzeilenformatierung
\pagestyle{fancy}
\renewcommand{\chaptermark}[1]{ \markboth{\MakeUppercase{\thechapter\; #1}}{} }
\renewcommand{\sectionmark}[1]{ \markright{\MakeUppercase{\thesection \; #1}} }
\fancyhf{}	% Alle Felder loeschen

% Abhaengig von Drucklayout
\ifthenelse{\boolean{@twoside}}{
	% Kopf- und Fußzeilen der Kapitelseiten
	\fancypagestyle{plain}{
   		\fancyhf{}
   		\renewcommand{\headrulewidth}{0pt}
   		\renewcommand{\footrulewidth}{0.75pt}
   		\fancyhead{}
   		\fancyfoot[OR]{\thepage}
	}
	
	% Kopf- und Fusszeile bei Nicht-Kapitelseiten
	\renewcommand{\headrulewidth}{0.75pt}
  \renewcommand{\footrulewidth}{0.75pt}
	\fancyfoot[EL,OR]{\thepage}
	\fancyhead[LO]{\scriptsize \rightmark}
	\fancyhead[RE]{\scriptsize \leftmark}
	
}{
}

%% Ueberschriftenformatierung
\titlespacing{\chapter}{0pt}{0pt}{75pt}
\titleformat
{\chapter} % command
[hang] % shape
{\bfseries\LARGE} % format
{\thechapter \ \ } % label
{0pt} % sep
{} % before-code
[] % after-code


%% Textformatierung
\renewcommand{\baselinestretch}{1.25}\normalsize	% Zeilenabstand
\frenchspacing 	% Ausschalten des Zusatzzwischenraums nach Satzzeichen

% Einstellung des Absatzes
\parskip1ex plus .2ex minus .2ex % zusaetzlicher Abstand
\parindent0pt % Einzug

\setcounter{secnumdepth}{3} % 3 Gliederungsebenen im laufenden Text

\sloppy % Laesst unguenstige Zeilenumbrueche bei schmalen Spalten zu


%% Labels und Referenzierung
\renewcommand\thesubfigure{ \alph{subfigure})} 


%% Nicht-nummerierte Listen
\setlist[itemize,1]{
	label={\color{fapsgruen}$\medblacksquare$}, % Label
	itemsep=5pt, % Abstand der Items innerhalb einer Ebene
	parsep=-5pt, % Abstand Paragraphen
	topsep=5pt, % Abstand Text/Items
	labelsep=5pt % Abstand Label/Text (bei Aenderung: leftmargin aus setlist[itemize,2])
}
\setlist[itemize,2]{
	label={\color{fapsgruen}$\smallblacksquare$}, % Label
	itemsep=-1pt, % Abstand der Items innerhalb einer Ebene
	parsep=0pt, % Abstand Paragraphen
	labelsep=3pt, % Abstand Label/Text
	leftmargin=9pt % Einzug (labelsep aus setlist[itemize,1] beachten)
}

%% Nummerierte Listen arabisch/roemisch
\newenvironment{einszweidrei}{\begin{enumerate}\renewcommand{\labelenumi}{\arabic{enumi}.}}{\end{enumerate}}
\newenvironment{iii}{\begin{enumerate}\renewcommand{\labelenumi}{\roman{enumi})}}{\end{enumerate}}



%% Formelausrichtung
%\ifthenelse{\equal{\Formelausrichtung}{links}}{
	%\setlength{\mathindent}{2cm}
%}{}

%% Formatierung figures und Co.
%\newlength{\matfigwidth}
%\setlength{\matfigwidth}{12cm}
%\newlength{\figheight}
%\setlength{\figheight}{10cm}
%\setlength{\abovecaptionskip}{10pt}
%
%\captionsetup{width=.95\textwidth}

\raggedbottom 