
% Makros

\newcommand{\TODO}[1]{{\color{red} #1}}
\newcommand{\EVTL}[1]{{\color{blue} #1}}


\newcommand{\VECSYM}[1]{\boldsymbol{#1}}			% Symbol als Vektor (fett)
\newcommand{\VEC}[1]{\mathbf{#1}}					% Vektor (fett)
\newcommand{\sign}[1]{\mbox{sign}\left\{#1\right\}}	% Signum-Funktion
\newcommand{\intd}{\textnormal{\;d}}				% d bei einem Integral
\newcommand{\ind}[1]{_\textnormal{#1}}				% konstanter Index in Matheumgebung (nicht kursiv)
\newcommand{\einheit}[1]{\,\textnormal{#1}}			% Einheit in Matheumgebung (nicht kursiv)
\newcommand{\const}[1]{\textnormal{#1}}				% Konstante in Matheumgebung (nicht kursiv)
\newcommand{\prozent}{\,\%}		% Prozent im Mathemodus


% Begriffe hervorheben
\newcommand{\significant}[1]{{\bf #1}}
\newcommand{\english}[1]{\emph{(engl. #1)}}

% Name
\newcommand{\name}[1]{\textsc{#1}}

% Abkuerzungs- und Symbolverzeichnis
\newcommand{\abk}[2]{({#1})\nomenclature[A]{#1}{#2}}
%\abk{Abkuerzung}{ausgeschrieben}

\newcommand{\sym}[3]{{#1}\nomenclature[S,#2]{#1}{#3}}		% Sortierung eingefuegt
%\sym{Symbol}{Sortierung}{Symbolbezeichnung}

% Referenzen fuer Formeln
\newcommand{\glg}[1]{Gleichung~\eqref{#1}}
% Referenzen fuer Abbildung
\newcommand{\abb}[1]{Abbildung~\ref{#1}}
% Referenzen fuer Tabellen
\newcommand{\tab}[1]{Tabelle~\ref{#1}}


% Farben
\definecolor{dunkelgrau}{rgb}{0.8,0.8,0.8}
\definecolor{hellgrau}{rgb}{0.9,0.9,0.9}
\definecolor{fapsgruen}{RGB}{151,193,57}
\definecolor{fapsblau}{RGB}{41,97,147}
\definecolor{fapsgraudunkel}{RGB}{95,95,95}


% Angaben fuer Literaturverzeichnis
\newcommand{\bblin}{in}
\newcommand{\bblvolume}{Vol.}
\newcommand{\bbledition}{Auflage}
\newcommand{\bbleditor}{Hrsg.}
\newcommand{\bbleditors}{Hrsg.}
\newcommand{\bblseries}{aus der Reihe}
\newcommand{\bblpp}{S.}
\newcommand{\bblaccess}{Aufgerufen am}

% Rechts- und linksbuendige Spalten mit fester Breite
\newcolumntype{R}[1]{>{\RaggedLeft\arraybackslash}p{#1}}
\newcolumntype{L}[1]{>{\RaggedRight\arraybackslash}p{#1}}
\newcolumntype{C}[1]{>{\centering\arraybackslash}p{#1}}

% Abstand der Punkte im Inhaltsverzeichnis / wenn keine Punkte, dann 0pc
\newcommand{\vzPunkte}{\titlerule*[1pc]{.}}

% Abstand zwischen Zeilen
\renewcommand{\arraystretch}{1.5}